To evaluate different feature engineering approaches the protein \textit{Acetylcholinesterase} was used.
The goal of this evaluation is to determine which feature engineering techniques shall be used on the
protein-ligand compounds. It is also of interest, which feature engineering techniques work best with each machine learning approach.
The \acrshort*[]{acc} measure is used to score the performance of the feature engineering method.
The metric is calculated using the validation data.

\subsection*{Neural network}
The following table is the result of applying the neural network on the datasets that where manipulated using feature engineering.

\begin{table}[H]
    \begin{center}
        \caption{Feature engineering validation accuracy \\ neural network}
        \begin{tabular}{lr}
            \toprule
            Name              & Validation Accuracy \\
            \midrule
            baseline\_nn      & 0.7801              \\
            fe\_smote\_nn     & 0.7801              \\
            fe\_pca\_nn       & 0.7589              \\
            fe\_rf\_mdi\_nn   & 0.7589              \\
            fe\_rf\_per\_nn   & 0.7518              \\
            fe\_nonhydrop\_nn & 0.7376              \\
            fe\_freq\_nn      & 0.7180              \\
            \bottomrule
        \end{tabular}
    \end{center}
\end{table}

The results state that the neural network approach does not improve greatly when applying the proposed feature engineering methods.
The \acrshort*[]{smote} method comes close to the performance of the baseline neural network. Therefore, it will be included for the analysis of the five complexes.

\subsection*{K nearest neighbor}
The following table is the result of applying the \acrshort*[]{knn} algorithm on the datasets that where manipulated using feature engineering.

\begin{table}[H]
    \begin{center}
        \caption{Feature engineering validation accuracy \\ k nearest neighbor}
        \begin{tabular}{lr}
            \toprule
            Name               & Validation Accuracy \\
            \midrule
            fe\_rf\_mdi\_knn   & 0.7934              \\
            fe\_rf\_per\_knn   & 0.7778              \\
            fe\_freq\_knn      & 0.7664              \\
            fe\_nonhydrop\_knn & 0.7550              \\
            fe\_pca\_knn       & 0.7550              \\
            baseline\_knn      & 0.7437              \\
            fe\_smote\_knn     & 0.7437              \\
            \bottomrule
        \end{tabular}
    \end{center}
\end{table}

The \acrshort*[]{knn} algorithm benefits greatly from the proposed feature engineering methods.
To contrast the baseline \acrshort*[]{knn} performance best, the feature engineering methods using random forest will be
evaluated for the protein-ligand complexes, as their performance supersedes the other methods.

\subsection*{Random forest}
The following table is the result of applying the random forest algorithm on the datasets that where manipulated using feature engineering.

\begin{table}[H]
    \begin{center}
        \caption{Feature engineering validation accuracy \\random forest}
        \begin{tabular}{lr}
            \toprule
            Name              & Validation Accuracy \\
            \midrule
            fe\_smote\_rf     & 0.8375              \\
            baseline\_rf      & 0.8362              \\
            fe\_rf\_mdi\_rf   & 0.8290              \\
            fe\_rf\_per\_rf   & 0.8221              \\
            fe\_freq\_rf      & 0.8107              \\
            fe\_nonhydrop\_rf & 0.8050              \\
            fe\_pca\_rf       & 0.8005              \\
            \bottomrule
        \end{tabular}
    \end{center}
\end{table}

The random forest algorithm does not benefit greatly from the proposed feature engineering methods.
Due to the more balanced dataset resulting from the \acrshort*[]{smote} method a slight performance boost can be observed.
Therefore, the \acrshort*[]{smote} algorithm will be applied to the remaining protein-ligand complexes.

\subsection*{Overall performance projections}
The following table lists all the feature engineering accuracies for all the machine learning approaches.

\begin{table}[H]
    \begin{center}
        \caption{Feature Engineering Validation Accuracy \\ overall}
        \begin{tabular}{lr}
            \toprule
            Name               & Validation Accuracy \\
            \midrule
            fe\_smote\_rf      & 0.8375              \\
            baseline\_rf       & 0.8362              \\
            fe\_rf\_mdi\_rf    & 0.8290              \\
            fe\_rf\_per\_rf    & 0.8221              \\
            fe\_freq\_rf       & 0.8107              \\
            fe\_nonhydrop\_rf  & 0.8050              \\
            fe\_pca\_rf        & 0.8005              \\
            fe\_rf\_mdi\_knn   & 0.7934              \\
            baseline\_nn       & 0.7801              \\
            fe\_smote\_nn      & 0.7801              \\
            fe\_rf\_per\_knn   & 0.7778              \\
            fe\_freq\_knn      & 0.7664              \\
            fe\_pca\_nn        & 0.7589              \\
            fe\_rf\_mdi\_nn    & 0.7589              \\
            fe\_nonhydrop\_knn & 0.7550              \\
            fe\_pca\_knn       & 0.7550              \\
            fe\_rf\_per\_nn    & 0.7518              \\
            baseline\_knn      & 0.7437              \\
            fe\_smote\_knn     & 0.7437              \\
            fe\_nonhydrop\_nn  & 0.7376              \\
            fe\_freq\_nn       & 0.7180              \\
            \bottomrule
        \end{tabular}
    \end{center}
\end{table}

As seen in the table the random forest approaches tend to yield the best result when applied to the validation portions of the datasets.
Therefore, this approach is very likely to score better on the test sets as well. 
