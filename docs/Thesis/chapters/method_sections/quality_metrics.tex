In order to make the results from this thesis comparable to the results from the 
scoring function introduced in \cite[]{Birklbauer2021} by Micha Birklbauer the same quality metrics have been 
implemented for this thesis. The following will provide an overview for the used metrics.
\subsection{Terminology} \label{terminology}
To calculate the metrics that are mentioned within this chapter the following base terms are necessary:
\begin{description}
    \item [TP] -- \textbf{T}rue \textbf{P}ositives are active samples, which are classified as such
    \item [TN] -- \textbf{T}rue \textbf{N}egatives are inactive samples, which are classified as such
    \item [FP] -- \textbf{F}alse \textbf{P}ositives are inactive samples, which are classified as active
    \item [FN] -- \textbf{F}alse \textbf{N}egatives are active samples, which are classified as inactive 
\end{description}

\subsection{Visual metrics}
For better visualization of the four base metrics mentioned in \href{terminology}{2.4.1}
this thesis displays the resulting data in a confusion matrix. This metric displays distribution of the results over the four base metrics.

In addition to that, the ROC(receiver operating characteristic) curve will also be displayed for the results.
The ROC curve is a collection of points in a two-dimensional 
space, where their location is defined by the FPR \href{fpr}{2.4.4} on the x-axis and the TPR($\frac{\#TP}{\#TP+\#FN}$) on the y-axis. 
Each point on this line depicts the ratio of FPR to TPR at a certain score cutoff. \cite[]{Lopes2017}


\subsection{Accuracy}
\acrfull*[]{acc} describes which portion of the predicted samples was accurately assigned to the correct class and is defined as follows:
\begin{equation*}
    \text{ACC} = \frac{\#TP+\#TN}{\#TP+\#TN+\#FP+\#FN}
\end{equation*}
\cite[]{Hossin2015}
\subsection{False positive Rate}\label{fpr}
\acrfull*[]{fpr} describes the compounds that were incorrectly classified as active in relation to all inactive compounds dand is defined as follows:
\begin{equation*}
    \text{FPR} = \frac{\#FP}{\#TN+\#FP}
\end{equation*}
\cite[]{Lopes2017}
\subsection{Area under the curve}
\acrfull*[]{auc} is a metric which stems from the ROC curve. 
The integral of the ROC curve is calculated using the \href{https://scikit-learn.org/stable/index.html}{scikit-learn
} package and is always between 0 and 1\cite[]{Lopes2017}.
\subsection{Yield of Actives}
\acrfull*[]{ya} describes the true positive compounds in relation to all as active labeled compounds and is defined as follows:
\begin{equation*}
    \text{Ya} = \frac{\#TP}{\#TP+\#FP}
\end{equation*}
\cite[]{Giordano2022}
\subsection{Enrichment Factor}
The \acrfull*[]{ef} describes the relation of the truly active compounds among all as active predicted complexes and the relative share of active compounds in the dataset.
This metric is defined as follows:
\begin{equation*}
    \text{EF} = \frac{\frac{\#TP}{\#TP+\#FP}}{\frac{\#TP+\#FN}{\#TP+\#TN+\#FP+\#FN}}
\end{equation*}
\cite[]{Lopes2017}  
\subsection{Relative Enrichment Factor}
The \acrfull*[]{ref} describes the relation of the EF to the maximum achievable EF. The REF is defined as follows:
\begin{equation*}
    \text{REF} = \frac{100 *\#TP}{\min(\#TP + \#FP,\#TP+\#FN)}
\end{equation*} 
\cite[]{Lopes2017}  

