The following chapter is dedicated to explaining the data used for this thesis.
This includes detailed descriptions of the protein-ligand complexes as well as their interaction types.

\subsection{Proteins}
The following five proteins have been used to conclude this thesis:
\subsubsection*{Acetylcholinesterase}
Acetylcholinesterase (AChE) is an efficient enzyme in the nervous system that breaks down 
acetylcholine (Ach), a messaging molecule, into choline and acetate. 
It's found in high concentrations at junctions between nerve cells and muscles.
AChE has various functions beyond just breaking down Ach, 
and it's present in both nerve and non-nerve tissues.
Because AChE is so important, some toxins like insecticides and nerve agents target it.
 This versatility of AChE makes it a key player in nervous system function
 and a potential target for drugs to treat diseases\cite[]{Tripathi2010}.
\subsubsection*{Cyclooxygenase 1}
Cyclooxygenase 1 (COX1) and its isoform Cyclooxygenase 2 (COX2) play a substantial role in 
synthesising various prostaglandins. Due to their linkage with inflammations and pain
COX molecules are often targeted by anti-inflammatory drugs. In contrast to COX2, COX1 is found in most tissues
across the body. In addition to that COX1 is largely attributed with homeostatic functions
such as hemostasis and gastric cytoprotection\cite[]{Rouzer2009}.
\subsubsection*{Dipeptidyl peptidase IV}
Dipeptidyl peptidase IV(DPP4) can be is partially responsible
for hydrolysis of a prolyl bond between two residues from the N-terminus.
DPP4 is present in several processes including metabolism and cancer biology.
Due to its role within the metabolism DPP4 inhibitory drugs have been successfully used in the treatment of diseases type two.
DPP4 also plays a substantial role in the diagnosis of certain types of cancer. 
In most cases DPP4 is up regulated near cancerous growth, therefore locally elevated DPP4 levels can be 
an indicator for cancer\cite[]{Yu2010}.
\subsubsection*{Monoamine oxidase B}
Monoamine oxidase B(MAOB) plays a major role in the breakdown of neurotransmitters(monoamines) within the body.
The compound is mainly expressed in glial-cells and platelets.
Its function categorizes MAOB as an important research compound, as MAOB inhibition has been proven to improve 
various neurological conditions. This stems from the fact that changes in the monoamine levels are
associated with a myriad of neurological problems\cite[]{Ramsay2016}.
\subsubsection*{Soluble epoxide hydrolase}
Soluble epoxide hydrolase (sEH) is part of an inflammatory pathway similar to COX.
It has been shown that inhibition of sEH reduces inflammation. In contrast to COX it does not 
completely disable the synthesis of pro-inflammatory compounds but rather balance their levels\cite[]{Schmelzer2005}.
\subsection{Interactions → ask Micha regarding grade of detail}
The following interactions have been chosen for this thesis:
\begin{itemize}
    \item hydrogen bonds
    \item water bridges
    \item salt bridges
    \item halogen bonds
    \item hydrophobic interactions
    \item pi-stacking
    \item pi-cation
\end{itemize}
\cite[]{Birklbauer2021}
\subsection{Data origin and structure}
The provided data is the result of the thesis \cite[]{Birklbauer2021} by Micha Birklbauer.
 