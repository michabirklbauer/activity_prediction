The following chapter is dedicated to explaining the data used for this thesis.
This includes detailed descriptions of the protein complexes as well as their interaction types.

\subsection{Proteins}
The following five proteins have been used as grounds for this thesis:
\subsubsection*{Acetylcholinesterase}
\acrfull*[]{ache} is an efficient enzyme in the nervous system that breaks down 
\acrfull*[]{ach}, a messaging molecule, into choline and acetate. 
It's found in high concentrations at junctions between nerve cells and muscles.
AChE has various functions beyond just breaking down Ach, 
and it's present in both nerve and non-nerve tissues.
Because AChE is so important, some toxins like insecticides and nerve agents target it.
 This versatility of AChE makes it a key player in nervous system function
 and a potential target for drugs to treat diseases\cite[]{Tripathi2010}.
\subsubsection*{Cyclooxygenase 1}
\acrfull*[]{cox1} and its isoform \acrfull*[]{cox2} play a substantial role in 
synthesising various prostaglandins. Due to their linkage with inflammations and pain
COX molecules are often targeted by anti-inflammatory drugs. In contrast to COX2, COX1 is found in most tissues
across the body. In addition to that COX1 is largely attributed with homeostatic functions
such as hemostasis and gastric cytoprotection\cite[]{Rouzer2009}.
\subsubsection*{Dipeptidyl peptidase IV}
\acrfull*[]{dpp4} protein is partially responsible
for hydrolysis of a prolyl bond between two residues from the N-terminus.
DPP4 is present in several processes including metabolism and cancer biology.
Due to its role within metabolism DPP4 inhibitory drugs have been successfully used in the treatment of diabetes type two.
DPP4 also plays a substantial role in the diagnosis of certain types of cancer. 
In most cases DPP4 is up regulated near cancerous growth, therefore locally elevated DPP4 levels can be 
an indicator for cancer\cite[]{Yu2010}.
\subsubsection*{Monoamine oxidase B}
\acrfull*[]{maob} plays a major role in the breakdown of neurotransmitters (monoamines) within the body.
The compound is mainly expressed in glial-cells and platelets.
Its function categorizes MAOB as an important research compound, as MAOB inhibition has been proven to improve 
various neurological conditions. This stems from the fact that changes in the monoamine levels are
associated with a myriad of neurological problems\cite[]{Ramsay2016}.
\subsubsection*{Soluble epoxide hydrolase}
Soluble epoxide hydrolase (sEH) is part of an inflammatory pathway similar to COX.
It has been shown that inhibition of sEH reduces inflammation. In contrast to COX it does not 
completely disable the synthesis of pro-inflammatory compounds but rather balance their levels\cite[]{Schmelzer2005}.
\subsection{Interactions}
Interactions define how proteins interact with each other or other types of ligands.
There are a lot of interactions which can be used for determining whether a certain compound might be considered active.
The following interactions have been used by the PLIP-Algorithm to produce the base data for this thesis:

\begin{table}[h!]
    \centering
\begin{tabular}{ | m{10em} | m{30em}| } 
    \hline
    \textbf{interaction} &\textbf{description\cite[]{Birklbauer2021}}
    \\
    \hline
    hydrogen bonds& A hydrogen bond is defined as the interaction between a 
    hydrogen atom, connected to a more electronegative atom, and another atom or molecule. \\
    \hline
    water bridges   &A water bridge occurs when the ligand and the protein
    both bind to a water molecule through hydrogen bonds.\\
    \hline
    salt bridges&  Salt bridges are ion pairs which stick together due to
    large difference in charge and the resulting electrostatic interaction. \\
    \hline
    halogen bonds&  Halogen bonds are defined as the interactions between the electrophilic region
    around a halogen atom and a nucleophilic region.\\
    \hline
    hydrophobic interactions& Aggregates formed as a result
    of a hydrophobic interaction between hydrocarbons in an 
    aqueous medium are called hydrophobic interactions.\\
    \hline
    pi-stacking   & Interactions between neighboring aromatic 
    rings are called pi-stacking. Due to the pi-electron density the ring is partially positively charged around 
    the periphery and negatively charged above both aromatic faces. As a result electrostatic forces build between aromatic rings,
     and they are attracted to one another.\\
    \hline
    pi-cation& Cations and pi-stacks who bind through electrostatic 
    forces at a pi-stacks face are called pi-cation interactions.\\
    \hline
   \end{tabular}
   \caption{interaction types}
\end{table}
    


\subsection{Data origin and structure} 
The provided data is a byproduct of the thesis \cite[]{Birklbauer2021} by Micha Birklbauer.
The interaction data was produced using the \textit{PLIP Algorithm} \cite[]{Salentin2015} 
on the aforementioned proteins.

The PLIP Algorithm consists of four major stages: 
\begin{description}
    \item[Structural Preparation] -- \textbf{SP} 
    \newline During the preparation step the input structure is hydrogenated and the ligands(including their binding sites) are extracted. 
    \item[Functional Characterization] -- \textbf{FC}
    \newline Using the structure of the complex a myriad of functional groups are detected. This includes binding site atoms, hydrophobic atoms and aromatic rings just to name a few.
    \item[Rule Based Matching] -- \textbf{RBM}
    \newline In the third step the algorithm investigates all interactions between the ligand and the protein, which can be attributed to geometric constraints. Hydrogen bonds are detected here.
    \item[Filtering of Interactions] -- \textbf{FoI}
    \newline This is a cleanup step where redundant or overlapping interactions get removed from the dataset. 
\end{description}
\newblock
The result of the PLIP Algorithm is a lineup of every interaction for each binding site and ligand\cite[]{Salentin2015}.
This data has been used as a basis for the machine learning approaches discussed in this thesis.