The following chapter summarizes the recent developments in drug design using various machine 
learning techniques. 

Today there exist a multitude of machine learning approaches in the field of drug design and activity prediction.
As a result of various AI breakthroughs in recent years there have been numerous research projects regarding the usability of artificial intelligence
in various bioinformatic domains.
One area where machine learning can be applied is quality assessment.
\textit{SVMQA} utilizes support vector machines to assess the quality of structural protein models\cite{Manavalan2017}.
Support Vector machines have also been used for the \textit{DeNovo} algorithm to detect protein-protein interactions\cite{Eid2016}.
AI has also been used to successfully identify drug responsive biomarkers in pre-clinical data using regression algorithms\cite{Li2015}.
In the field of synthesis-prediction AI has largely replaced the rule- and heuristic-based systems in place since the 1960s\cite{Johansson2019}.

Due to developments in the field of deep learning, this technology has found numerous applications in biochemistry\cite{Chen2018}.
One of which is \textit{deepDTnet}, which is a deep learning based algorithm used to identify new targets and repurpose existing drugs in a drug-gene-disease environment.
This is done by embedding already existing interaction profiles into low dimensional vector spaces\cite{Zeng}.
Deep learning also has its applications in the classification and segmentation of microscopic imagery\cite{Kraus2016}.
\textit{MILCDock} uses the Output of five traditional Scoring Functions as input for a neural Network. 
The input for the neural network comes from the tools \textit{LeDock}, \textit{Autodock Vina}, \textit{PLANTS},  \textit{Autodock4}, and \textit{rDock}. 
This technique has a slight performance benefit when compared to traditional scoring functions \cite[]{Morris2022}.
