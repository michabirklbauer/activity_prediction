\chapter{Conclusion}
The goals for this thesis where twofold. Firstly, the applicability of numerous machine learning approaches for activity prediction, and
secondly the application of feature engineering methods to the base datasets.
\\\\
The first goal was addressed by applying \textit{k nearest neighbor}, \textit{random forest} and \textit{neural network} algorithms.
The results from those baseline implementations are very promising. The second goal builds upon the first goal. Through the implementation of various feature engineering 
methods the results of the baseline machine learning algorithms improved for a select number of configurations. An especially noteworthy feature engineering method in this context is
\acrshort*[]{smote}. Through the balancing of the provided datasets using \acrshort*[]{smote} the accuracy of all models improved.
When comparing the different machine learning approaches the \textit{random forest} algorithm was able to achieve the best performance overall.
The source code for this thesis can be found at \href{https://github.com/michabirklbauer/activity_prediction}{https://github.com/michabirklbauer/activity\_prediction}. 