\chapter{Kurzfassung}

\begin{german}
Der Prozess der Entdeckung von Medikamenten oder chemisch aktiven Verbindungen ist zeitaufwendig
und daher auch sehr teuer. Aus diesem Grund ist es von allgemeinem Interesse, die Effizienz der
Medikamentenforschung zu verbessern und die Gesamtkosten dieses Prozesses zu senken. Die Identifizierung
neuer Medikamente oder Wirkstoffkandidaten kann mit einem \textit{in-vitro} Ansatz, dem sogenannten \textit{high-throughput screening},
oder mit einem computergestützten Ansatz, dem \textit{virtual screening}, erreicht werden. Diese Arbeit basiert auf
"Automatic identification of important interaction-
sand interaction-frequency-based scoring in protein-ligand complexes"
von Micha Birklbauer\cite[]{Birklbauer2021} und schlägt einen neuartigen \textit{virtual screening}-Ansatz
zur Evaluierung von Protein-Docking im Medikamentendesign unter Verwendung verschiedener Machine-Learning-Methoden vor.

Zu diesem Zweck wurden die Algorithmen \textit{k nearest neighbor}, \textit{random forest} und ein künstliches neuronales Netzwerk implementiert.
Die zugrunde liegenden Daten für die Algorithmen bestehen aus Protein-Liganden-Komplexen von fünf medizinisch relevanten Proteinen mit Wirkstoffen 
und Molekülen aus der \textit{DUD-E}-Datenbank\cite[]{Mysinger2012}. Zur Extraktion der Interaktionsdaten wurde der \textit{PLIP}-Algorithmus\cite[]{Salentin2015} verwendet.
Zusätzlich zu den Machine-Learning-Algorithmen wurden auch verschiedene Feature-Engineering-Methoden implementiert.

Beim Vergleich der Leistung der drei Algorithmen unter der Verwendung verschiedener Feature-Engineering-Techniken
schneidet der Random-Forest-Klassifikator in Bezug auf alle implementierten Metriken am besten ab.
Die implementierten Algorithmen können genauere Klassifikationsergebnisse erzielen als \cite[]{Birklbauer2021}.
Die Ergebnisse der Machine-Learning-Algorithmen, welche im Rahmen dieser Arbeit entwickelt wurden, liefern im Durchschnitt
um 5-10\% genauere Ergebnisse. Darüber hinaus erkennen die Algorithmen auch weniger falsch positive Proteinkomplexe.
Dies ist ein entscheidender Vorteil, da weniger Zeit mit der Untersuchung inaktiver Protein-Ligand Komplexe
im Labor verbracht wird.

Diese Arbeit zeigt die Anwendbarkeit von Machine Learning für das interaktions-basierte virtuelle Screening von Protein-Liganden-Komplexen.
Die Qualität der genannten Methoden
kann eventuell durch das weitere Anpassen der Algorithmen noch verbessert werden.

Der Quellcode, welcher für diese Arbeit implementiert wurde, ist auf \\\href{https://github.com/michabirklbauer/activity_prediction}{https://github.com/michabirklbauer/activity\_prediction} 
zu finden.
\end{german}
